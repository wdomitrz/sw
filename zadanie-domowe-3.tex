%!TEX encoding =  UTF-8 Unicode

\documentclass[final,12pt]{article}
\usepackage{a4wide}
\usepackage{amsmath}
\usepackage{amssymb}
\usepackage{latexsym}
\usepackage[polish,english]{babel}
\usepackage[T1]{fontenc}
\usepackage[utf8]{inputenc}

\usepackage[pdftex]{color}
\usepackage[final]{listings}

%%%%%%%%%%%%% listings
\lstdefinelanguage{whileprograms}{morekeywords={while,do,if,then,else,),(,decr,in,wrt},%
   sensitive,%
   morecomment=[l]//,%
   morecomment=[s]{\{}{\}},%
   morecomment=[s]{[}{]},%
    basicstyle=\small\tt,
    keywordstyle=\normalfont\bfseries\color{black},
    commentstyle=\color{blue},
    mathescape=true,
}

\lstset{language=whileprograms,flexiblecolumns=true,mathescape=true,frame=none}
\lstset{commentstyle=\it,basicstyle=\tt}
\lstset{literate={<=}{{$\leq$}}2
                {>=}{{$\geq$}}2
                 {^}{{$\land$}}2
                 {&}{{$\land$}}2}

\pagestyle{empty}

\begin{document}
\lstset{language=whileprograms} \selectlanguage{polish}

\begin{center}
{\bf Semantyka i weryfikacja programów 2019/20.\\
     Zadanie domowe nr 3 }

\end{center}
Dany jest następujący program w języku TINY, obliczający 
najmniejszą wspólną wielokrotność liczb $x$ i $y$.
\begin{lstlisting}
{x>0, y>0}
xx:=x; yy:=y;
wx:=x; wy:=y;
{                                                                                      }
while {                                                                                }
   wx<>wy do
      {                                                                                }
      if wx<wy then
            while {                                                                    }
               wx+2*xx<=wy do
                  {                                                                    }
                  xx:=2*xx;
            {                                                                          }
            wx:=wx+xx;
            xx:=x
      else
            while {                                                                    }
               wy+2*yy<=wx do
                  {                                                                    }
                  yy:=2*yy;
            {                                                                          }
            wy:=wy+yy;
            yy:=y
{                                                                                      }
{wx=NWW(x,y)}
\end{lstlisting}
Udowodnij częściową poprawność programu względem podanej
specyfikacji
tj.
\begin{enumerate}
\item Podaj niezmienniki wszystkich pętli.
\item Wstaw odpowiednie formuły w nawiasy klamrowe tak, aby powstałe anotacje umożliwiały przeprowadzenie dowodu częściowej poprawności.
\end{enumerate}

W swoich formułach możesz użyć dwuargumentowej funkcji {\tt NWW}, oznaczającej najmniejszą wspólną wielokrotność, oraz dwuargumentowego predykatu {\tt div}, oznaczającego podzielność. Można je zresztą łatwo zdefiniować w logice pierwszego rzędu, pisząc:
\[
	{\tt div}(x,y) \iff \exists z. (z>0\land y=x\cdot z)
\]
a zamiast ${\tt NWW}(x,y)=z$ pisząc:
\[
	{\tt div}(x,z) \land {\tt div}(y,z) \land \forall u.(({\tt div}(x,u) \land {\tt div}(y,u)) \implies z\leq u)
\]

\end{document}
